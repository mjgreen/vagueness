In most academic use, the word `vagueness' has a specific meaning. Keefe and Smith, for example, state ``vague predicates have borderline cases, have fuzzy boundaries, and are susceptible to sorites paradoxes'' \citep[p.~4]{keefe1997vagueness}, also \citet{EgreKlinedinst}.  The crucial criterion is the existence of borderline cases: ``a word is precise if it describes a well-defined set of objects. By contrast, a word is vague if it is not precise'' \citet[p.~1]{lipmanvague}. A typical example is the word ``tall'', as applied to people for example, because here is no precise, known height which separates those who are tall from those who are not. The crucial point is that ``tall'' admits borderline cases (i.e., people who may or may not count as tall), which are the hallmark of vagueness as we use the term.

Linguists, philosophers of language, and more recently game theorists, have asked why natural languages contain so many vague expressions \citet{Lipman:2000fk, lipmanvague}, which are
often used even in situations where the speaker could have used an expression that is not vague (i.e., crisp); in these situations we say that vagueness is used \emph{strategically}. By introducing borderline cases, these expressions create potential misunderstandings, thereby creating ``a worldwide several-thousand year efficiency loss'' \citet[][p.~1]{lipmanvague}. Lipman explains the point by means of a scenario in which a speaker describes a person to a hearer, who needs to identify that person in the arrivals hall of of an airport. In such a scenario, a precise description of the person's height (e.g., ``The person's height is 187.96 cm'') would be more useful than a vague one (``The person is tall''). Lipman uses this scenario to explain why standard game theory models of communication \citep[e.g.,][]{Crawford:1982lr} predict that, under certain conditions, a crisp act of communication will always have more utility than a vague act that communicates the same state of affairs. 

Lipman argued that the efficiency loss resulting from vague expressions would be unlikely to have arisen unless there are advantages as well as disadvantages associated with vague expressions. Lipman asked, essentially, what these advantages might be. Several tentative answers to Lipman's question have been offered \citep[see][]{van2009utility, vanDeemterBook}. One of the most promising answers appears to be the idea that vague expressions are easier to process, by a speaker and/or a hearer, than expressions that are not vague (i.e., crisp) \citep[e.g.,][]{lipmanvague,De-Jaegher:2003lr,vanrooij2003lr}. For example, \citeauthor{lipmanvague} writes: ``For the listener, information which is too specific may require more effort to analyze'' (\citeyear[][p.~11]{lipmanvague}). We shall refer to this as the \emph{cost reduction} hypothesis. 

This article brings an experimental approach to these issues, focussing on vagueness in descriptions (e.g., ``the square with few dots'') and its effect on the hearer's ability to act on a given description, as measured by the time that it takes hearers to click on the referent of a description\footnote{Other metrics could have been chosen, such as hearers' ability to remember information, for example, or error rates. Although error rates play a minor role in the present paper, for reasons that will become clear, we focus on response times in particular.}. We find that, although hearer benefits from vague descriptions are straightforward to demonstrate in many cases, a closer experimental analysis militates against the conclusion that vagueness itself -- as defined above, in terms of the existence of borderline cases -- lies at the heart of these results. Instead, it is other factors, such as the presence of an overt numerical expression in the description, that proved to be decisive. We believe that, despite the fact that our experiments are unavoidably focussed only on a specific class of vague expressions (since any experiment can only deal with a limited number of different stimuli), these findings are potentially important, because they call into question whether ``strategic'' vagueness (i.e., vagueness where the speaker had a choice, because she could have been produced a crisp expression instead) has any advantages at all. In other words, returning to Lipman's question, it is possible that vagueness has evolved partly as a necessary evil (e.g., because of the limits of observation and prediction) and partly as a side effect of other factors.
