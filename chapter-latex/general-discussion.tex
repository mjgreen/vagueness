To summarise our findings, we asked why strategic vagueness is as frequent as it is and we decided to focus on what we see as the most promising explanation, namely that vague expressions are easy to process by speakers and hearers: the cost hypothesis, as we have called it. We decided to test this hypothesis by experimentally investigating whether vague descriptions are resolved by hearers more quickly than crisp ones. Although we were able to find some interesting (and statistically highly significant) effects, it appears to us that if our sequence of experiments is assessed as a whole, it cannot be seen as confirmation of the cost hypothesis.

To explain why this is, let us summarise our findings so far: %
%\citet{green2013utility} showed that responses were faster and more accurate when the instructions were vague than when they were crisp, but the experiment was unable to distinguish effects of vagueness from those of number-avoidance or selection task: the vague conditions were also in verbal rather than numerical format, and they mandated a comparison strategy rather than a matching strategy. 
%
Experiment 1 showed that number avoidance in the verbal format instructions is an important factor driving the faster response times in the task, and that vagueness does not have any additional benefit in either the verbal format instructions or the numerical format instructions. However, Experiment 1 did not distinguish benefits of number avoidance from benefits of the comparison selection task. In Experiments 2 and 3 we manipulated vagueness and the selection task, separately at each level of numerical format. Across the two experiments, we found that the comparison-task instructions attracted faster response times than the matching-task instructions. Within the two experiments we found that vagueness exerts benefits when the selection task is \emph{comparison}, but not when the task is \emph{matching}.

What is one entitled to conclude? Given that we were able to identify a class of situations in which vague expressions led to faster response times than crisp ones, would it be valid to conclude that we have finally discovered an advantage for vagueness that cannot be ascribed to some other factor? We believe the answer to this question is negative.

\begin{table}[htbp]
\caption{Vagueness as range reduction: a summary of Experiments 2 and 3}
\label{Vagueness as range reduction}
\centering
\begin{tabular}{cccc}
\hline\noalign{\smallskip}
selection task 					& vagueness		& candidates	& effect of vagueness						\\
\noalign{\smallskip}\hline\noalign{\smallskip}
\multirow{ 2}{*}{comparison} 	& crisp 		& 2				& \multirow{ 2}{*}{vagueness advantage}  	\\
\noalign{\smallskip}\cline{2-3}\noalign{\smallskip}
								& vague			& 1				&                                           \\
\noalign{\smallskip}\hline\noalign{\smallskip}
\multirow{ 2}{*}{matching} 		& crisp 		& 1 			& \multirow{ 2}{*}{vagueness disadvantage}	\\
\noalign{\smallskip}\cline{2-3}\noalign{\smallskip}
								& vague			& 2				&								 			\\
\noalign{\smallskip}\hline
\end{tabular}
\end{table}

To see why, consider Figures \ref{resultsD-exp-2} and \ref{resultsE-exp-3}. Both figures depict four conditions, depending on whether the expression was crisp or vague, and depending on whether the referent could be identified using a comparison strategy or not. Two of the resulting four conditions result in an expression that can denote either of two referents; the other two conditions result in an expression that can only denote one referent, with the other possible referent being a marginal candidate at best.

To see why vagueness has opposite effects, depending on whether it is used in matching or comparison situations, consider the stimulus with (6,15,24) dots. Now compare `Choose a square with 6 dots' with its vague counterpart `Choose a square with about 10 dots': by adding the word `about', we broaden the range of squares that the expression might be referring to. On the other hand, compare `Choose a square with fewer than 20 dots' with `Choose a square with far fewer than 20 dots': by adding the word `far', we did not broaden the range of squares denotable by the expression: we narrow it down, because only some of the squares that have fewer dots may have {\em far} fewer dots.

The benefits of vagueness in the \emph{comparison} task in experiments 3 and 4 could thus be explained as differences in the number of valid targets for the expression. This leads us to speculate that the benefit for vagueness here could be due to the vague expression foregrounding a particular valid target while the crisp expression carries with it the additional task of distinguishing between two alternative valid targets, something we propose to call a ``range-reduction'' benefit.

The observation that conditions with 1 candidate lead to shorter response times than conditions with 2 candidates is consistent with the range reduction hypothesis, but not with the idea that vagueness is beneficial. It appears, in other words, that shorter response times will only result from a vague expression if this expression leads to range reduction. Once again, it is not vagueness itself that has advantages but a phenomenon (namely range reduction) that is an automatic concomitant of vagueness in some types of situations.\\[2ex]
%
\noindent Looking at the entire series of experiments, our findings suggest that the observed benefits of vague expressions may be due to factors other than vagueness: factors like avoiding numbers; permitting comparison tasks; and range reduction. The picture that is starting to emerge is subtle: on the one hand, in the situations that we have been studying vagueness is not intrinsically beneficial. On the other hand, vague expressions frequently possess other features that \emph{are} beneficial, and these are what give us the incorrect impression that vagueness itself is beneficial. Vagueness may thus have acquired a reputation that it does not deserve. The answer to Lipman's question, of why vagueness permeates human language (see our Introduction), may lie in a different direction after all, possibly relating to benefits for the speaker rather than the hearer.

A comparison may clarify the logic of the situation. In recent years a number of studies, focussing on red wine, have suggested that alcohol, consumed in low doses, may have health benefits. An alternative explanation, however, asserts that it is not the alcohol in the wine that was beneficial, but antioxidants from grapes. If this alternative explanation is correct, then alcohol may not be healthy after all.\\[2ex]
%
% KvD Feb 2017. I've diminished the role of NLG in the coming paragraph.
Our findings suggest a re-think of the questions on which much research on the utility of vagueness rests. Years of research on the logic of vagueness -- giving rise to such techniques as Partial Logic \citep[e.g.,][]{Fine}, Probabilistic Logic \citep{Edgington}, and Fuzzy Logic \citep{Zadeh} -- have primed the research community to expect that some special utility of vagueness is an important part of the answer, but our findings call this expectation into question. 

Although our own studies in this article have focussed on vagueness in descriptive Noun Phrases only, it seems to us plausible that vagueness plays a similar role in other
linguistic constructs. For example, consider reports on air temperature. Given a numerical
temperature measurement or prediction, we might word it as 
%
\begin{quote}
(a) {\em 27.2 degrees Celsius}, or\\
(b) {\em approximately 27 degrees}, or\\
(c) {\em above 25 degrees}, or\\
(d) {\em warm}, 
\end{quote}
%
among other candidate expressions. Which of these descriptions is most effective, for example as part of a weather report? If the linguistic literature is to be believed, then options (a) and (c) convey crisp information, whereas (b) and (d) are vague (i.e., they permit borderline cases). In the situations studied in our own experiments and the ones discussed in section 1, we found no evidence that vagueness is beneficial for hearers. Rather than asking whether a candidate expression is vague, other questions might shed more light on the choice, similar to the ones identified in our studies. These questions might focus on the amount of information that a given expression conveys (i.e., on granularity), on the avoidance of numbers, and on the use of evaluative terms. Let's see how this might pan out for the above examples from the weather domain.

First, the experiments by \citeauthor{Mishra01042011} suggest that it is important how much information is conveyed by an expression, and their findings are echoed by our own thoughts about range reduction (following Experiments 2 and 3). In the case of (a)--(d) above, it appears that (a) conveys the most detailed information (designating the smallest segment of the temperature scale), followed by (b), then (d), then (c) (e.g., 40 degrees is above 25, but at 40 Celsius the word ``warm'' is likely to give way to ``hot'' or ``scorching''): 
%
\begin{quote}
$a < b < d < c$
\end{quote}
%
If these hunches are correct, then it seems to us that it is relatively unimportant whether a given expression is vague or crisp. Other factors seem more important; moreover, it may depend on the task and the audience which of a-d is preferred. For example, an \emph{expert} may prefer to read expression (a), because it gives her the most detailed information on which to base her decisions. On the other hand, expression (d) (``warm'') is shorter than the other three and avoids the use of numbers; our experiments suggest that this may make (d) more rapidly understood than its competitors; earlier experiments point in the same direction, given the evaluative nature of ``warm'' (recall our section 1.2), which is especially important if the hearer is unfamiliar with the metric used. These considerations suggest that {\em non-experts} might prefer expression (d). 

One way to see why vagueness (as defined in our Introduction) may not benefit human communication is the following thought experiment. Suppose a group of speakers understand the word ``warm'' as vague, agreeing that temperatures above 26 count as warm, and temperatures below 24 do not count as warm, but considering temperatures between 24 and 26 as borderline cases. Now one day these speakers agree to sharpen up their definition deciding that, henceforth, ``warm'' means ``$>25$ degrees'' (as in (c) above): this decision resolves the borderline cases, while everything else remains the same. It seems unlikely that this change in language use, from a vague meaning to a crisp one (i.e., one that has no borderline cases anymore), would lower the utility of the word. Our experimental findings,
and the conclusions that we draw from them, are consistent with this idea.
