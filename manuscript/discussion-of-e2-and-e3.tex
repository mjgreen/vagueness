The main aim of these two experiments was to test whether vagueness confers any cognitive benefits over and above those due to differences in the selection task according to whether the instruction mandates a \emph{comparison} selection task or a \emph{matching} selection task, when number-use is held constant. The main effect of selection task showed that the assumption that the \emph{comparison} task is easier than the \emph{matching} task is well-founded. In both experiments people were reliably quicker to respond in the \emph{comparison} task. 

Vagueness, which was the phenomenon on which our investigation focussed, did not exert a significant main effect in response time. However when the comparison and selection tasks were analysed separately, there was small significant advantage for vagueness in the \emph{comparison} tasks, but a small significant disadvantage for vagueness in the \emph{matching} tasks. 
