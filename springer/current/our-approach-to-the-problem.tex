Are vague expressions processed more easily by readers than crisp ones? Like Lipman, we focus on situations where numerical information is used in order to identify a referent. Reference, in other words, will be the communicative task on which we focus, partly because of the interest that this topic has recently drawn in various areas of Cognitive Science \citep{vanDeemterCMR}. By looking at one specific type of vagueness, we will be able to investigate the costs and benefits of vagueness relatively thoroughly. Whether our findings generalise to other uses of vagueness is a question on which we will speculate in the final section of this chapter.

We have chosen a narrative strategy in which we address a sequence of four experiments with human reader chronologically, explaining how each experiment helped us refine our research question. In order to do justice to our findings, we need to describe these experiments in a fair amount of detail. 

Let us start by explaining the task that was given to the participants in our experiments. We used a {\em speeded forced choice} task to compare the processing costs of different references to quantities. In this context, speed and accuracy of responses are the key dimensions on which the different references can be compared. Each stimulus in the experiments was a set of dot arrays containing various number of dots, together with a preceding instruction (in the form of a referring expression) to choose one of the arrays with respect to its cardinality. The participant was asked to respond as quickly as possible while avoiding errors. We manipulated the instructions and the arrays in several ways across the four experiments. 

All the experiments shared the following properties: Stimuli were created using the language GNU Octave \citep{eaton:2002} and the Psychophysics Toolbox extensions \citep{ptbx1, ptbx2}. The position of the dots was randomised per-trial. The order in which trials were presented was randomised per-participant. There were 256 trials, presented in 4 blocks of 64 each, between which the participant could rest. A MacBook Pro laptop computer with a 13 inch screen presented the stimuli to the participants and recorded responses. Participants were recruited using email lists at the University of Aberdeen, and paid ten pounds for participating. All participants self-reported fluency in English, and had normal, or corrected-to-normal vision. The experiment was conducted in a quiet room. Participants were asked to respond as quickly as possible while avoiding errors. There was a block of practice trials after which participants could ask any questions, following which the experimenter left the room. All $p$ values reported for linear models were calculated using the R package \emph{lmerTest} \citep{lmerTest}. The complete data and analysis for this experiment and the others in this paper are available at \texttt{https://github.com/mjgreen/vagueness}.
%\texttt{\url{http://homepages.abdn.ac.uk/k.vdeemter/pages/vagueness-data-analysis/e1.html}
 
When the distance grows between two numbers, they become more easily distinguishable: the \emph{numerical distance effect} has been shown for comparing the cardinality of two sets of dots \citep{van123} and for processing Arabic numerals and number words \citep{Dehaene1996}. We manipulated the number of dots in each array such that some sets of arrays had smaller numerical distances and others had larger numerical distances. Where a number was mentioned in the instructions, it was always in the form of an Arabic numeral. When two numbers are presented with the smaller on the left, this left-side presentation facilitates responses indicating the smaller number: the \emph{Spatial-Numerical Association of Response Codes (SNARC)} effect \citep{dehaene1993mental, gevers2006automatic}. We controlled which side the smaller number appeared on to avoid systematic influences of this effect. 

There is abundant evidence \citep[e.g.,][]{trick1994small} that very small (i.e., \emph{subitizable}) quantities are recognised and processed by a distinct psychological mechanism that differs from that used to process larger quantities. We performed a pilot experiment \citep{green2011costreduction} in which we were able to confirm this finding in the experimental settings on which we are focussing in this paper. We found that, when participants were confronted with a stimulus consisting of two squares containing different numbers of dots\footnote{Such a stimulus is referred to hereafter as consisting of a set of {\em dot arrays}. The number of dots in an array is referred to as its cardinality. The physical arrangement of dots in each array is irregular.}, instructions of the form \emph{Choose the square with \emph{n} dots} led to consistently faster response times than instructions of the form \emph{Choose the square with many/few dots} when $2 \leq n \leq 5$; the converse was true for $n>5$. Given these findings, we henceforth focussed our studies on non-subitizable numbers, because it is there that vagueness is expected to have benefits.
